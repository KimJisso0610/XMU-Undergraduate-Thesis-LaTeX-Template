\chapter{摘\quad 要}

    论文的摘要是对论文研究内容和成果的高度概括。摘要应对论文所研究的问题及其研究目的进行描述,
    对研究方法和过程进行简单介绍,对研究成果和所得结论进行概括。\par
    摘要应具有独立性和自明性,其内容应包含与论文全文同等量的主要信息。
    使读者即使不阅读全文,通过摘要就能了解论文的总体内容和主要成果。\par
    论文摘要的书写应力求精确、简明。切忌写成对论文书写内容进行提要的形式,
    尤其要避免“第1章……;第2章……;……”这种或类似的陈述方式。\par
    示例:\par

    近几十年来,图像加密作为重要的信息安全领域之一,吸引了众多研究人员和科学家。
    然而,已经用不同的方法进行了几项研究,并且已经提出了新颖且有用的算法来改进安全图像加密方案。
    如今,混沌方法已在多个领域中被发现,例如密码系统的设计和图像加密。
    基于混沌方法的数字图像加密是一种新颖的图像加密方法。
    该技术采用随机混沌序列对图像进行加密,是一种高度安全、快速的图像加密方法。
    有限的准确性是这种技术的缺点之一。本文通过研究混沌序列和小波变换值来寻找差距。
    因此,提出了一种用于数字图像加密的新技术并改进了以前的算法。
    该技术在 MATLAB 中运行,并根据像素数变化率 (NPCR)、峰值信噪比 (PSNR)、
    相关系数和统一平均变化强度 (UACI) 等各种性能指标进行比较。
    仿真和理论分析表明了该方案的有效性,表明该技术是实际图像加密的合适选择。\\
    \\ 
    {\bfseries 关键词:}数字图像加密;图像处理;混沌随机序列;离散小波变换

