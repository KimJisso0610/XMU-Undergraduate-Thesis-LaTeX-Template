\chapter{Abstract}

    In recent decades, image encryption, as one of the significant information security fields, 
    has attracted many researchers and scientists. However, 
    several studies have been performed with different methods, 
    and novel and useful algorithms have been suggested to improve secure image 
    encryption schemes. Nowadays, chaotic methods have been found in diverse fields, 
    such as the design of cryptosystems and image encryption. 
    Chaotic methods-based digital image encryptions are a novel image encryption method. 
    This technique uses random chaos sequences for encrypting images, 
    and it is a highly-secured and fast method for image encryption. 
    Limited accuracy is one of the disadvantages of this technique. 
    This paper researches the chaos sequence and wavelet transform value to find gaps. 
    Thus, a novel technique was proposed for digital image encryption and improved previous algorithms. 
    The technique is run in MATLAB, and a comparison is made in terms of various performance 
    metrics such as the Number of Pixels Change Rate (NPCR), Peak Signal to Noise Ratio (PSNR), 
    Correlation coefficient, and Unified Average Changing Intensity (UACI). 
    The simulation and theoretical analysis indicate the proposed scheme's effectiveness and show 
    that this technique is a suitable choice for actual image encryption.\\
    \\
    {\bfseries Keywords:} digital image encryption; image processing; chaos random sequence; discrete wavelet transform

