\chapter{结论}

最近,已经提出了各种基于混沌的图像密码系统。目前的工作是利用混沌映射和小波变换的特征来处理基于混沌的算法
该算法的加密过程包括两个阶段。首先,我们进行了图像扩散操作。此外,通过执行小波变换,由于超混沌序列而大大减少
了混淆计算量。标准度量的仿真结果表明,所提出的算法对密钥具有高度的依赖性。该算法包括一个不错的加密效果。
此外,它可以抵抗噪音和减少攻击。 Y. Pourasad 等人已经从基准 MATLAB 测试图像中测试了
Lena、Peppers、Barbara、Baboon 和 Boat 图像的提出方法。此外,还描述了输入图像和加密图像的直方图。
此外,还记录了PSNR、NPCR、UACI和NC等加密性能分析标准。根据结果​​,Lena、Peppers、Barbara、Baboon 
和 Boat 的相关值分别为 95.48\%、99.64\%、98.09\%、91.37\% 和 90.01\%。为了评估所提出的方法的鲁棒性,
测试图像针对四种类型的图像处理攻击进行了测试:旋转、高斯噪声、中值过滤和直方图均衡。结果表明,所提出的设
计具有更高的鲁棒性和归一化相关性。根据结果​​,输入攻击不影响图像加解密。关于不同类型图像的 NC 值,中值滤波、旋
转和高斯噪声具有更高的 NC 值。这意味着所提出的方法的鲁棒性可以抵抗这些类型的攻击。然而,直方图均衡的影响是显著的。