\chapter{算法}

本节可按小标题划分。
它应该对实验结果、它们的解释以及可以得出的实验结论提供简明准确的描述。
实施建议算法的步骤是:\\ \par

\noindent
\hangafter=1
\setlength{\hangindent}{4em}
步骤1:排列一张灰度图。图像的大小设置为$m\times n$。 
此外,放置了数据矩阵$R$。 通过评估两个逻辑图,生成一个混沌序列。 与主图像进行 XNOR,扩散终止。

\noindent
\hangafter=1
\setlength{\hangindent}{4em}
步骤2:本步骤中,对步骤1中的漫反射图像进行小波分解,提取小波系数,记为ca1。

\noindent
\hangafter=1
\setlength{\hangindent}{4em}
步骤3:利用二维超混沌图CML产生混沌序列,并利用步骤2中建立的ca1进行位置混淆。

\noindent
\hangafter=1
\setlength{\hangindent}{4em}
步骤4:在最后一步,可以通过小波重建混淆图像,得到了加密的图像。加密的逆运算称为解密算法。 图像加密和图像解密中的系统参数和混沌序列的初值是一致的。

\section{加密评估指标}

我们通过选择一些基本参数来评估算法来衡量我们的密码方案的性能。视觉检查是评估加密图像的主要参数之一。
特征扩散调查是判断随机化算法的另一个参数。通过检查,确定产品与一组确定的特征的偏差。人工操作通常完成
检查;然而,机器视觉被用于自动化这个过程。由于算法的良好扩散,原始图像和加密图像之间的关联变得过于复
杂,无法简单地预测。在这里,我们研究了峰值信噪比 (PSNR) 计算指标,即加密图像和关键图像之间的关联。
最终,我们通过计算统一平均变化强度 (UACI) 和像素数变化率 (NPCR) 两个参数来评估规范扩散。\citep{ul2020algebra}
\section{峰值信噪比 (PSNR)}

峰值信噪比 (PSNR) 是通过均方误差 (MSE) 确定的工程公式,它通常用于图像质量评估。公式如下\citep{xiao2009analysis}:
\begin{equation}
    \text{PSNR}=10\log \left(\frac{255^2}{\text{MSE}(f,f')}\right)
\end{equation}
其中$f(x,y)$和$f'(x,y)$表示原始图像和重建图像的像素值。
\section{像素数变化率 (NPCR)}

扩散以判断加密算法随机化的最基本参数的个数来表示。 
NPCRs 用于检查图像加密算法的安全性。
考虑$C_1$和$C_2$作为两个具有大小的图像,我们定义了一个与图像大小相似的数组:
\begin{equation}
    D(i,j)= \genfrac{\{}{}{0pt}{}{0,\, \text{if}\, C_1(i,j)=C_2(i,j)}{1,\, \text{if}\, C_1(i,j)\neq C_2(i,j)}
\end{equation}
NPCR 确定两个不同图像中像素的百分比,其计算公式如下\citep{wang2020image, wang2016novel, yun2009digital}:
\begin{equation}
    \text{NCPR}=\frac{\sum_{i}^{} \sum_{j}^{} D(i,j) }{N\times M}\times 100\%
\end{equation}
\section{统一平均变化强度 (UACI)}

UACI 使用以下表达式确定两个加密图像 ($C_1$和$C_2$) 内差异的平均强度,用于评估加密方法的强度,
它的值基于图像的格式和大小。 
通过 UACI,可以评估加密图像和原始图像之间的平均强度变化。 
最大的 UACI 表明所建议的技术对各种攻击具有抵抗力。 
UACI 的确定如下(假设灰度图像的大小为$M\times N$)\citep{ying2004digital}:
\begin{equation}
    \text{UACI}=\frac{1}{N\times M}\left[\sum_{i}\sum_{j}\frac{C_1(i,j)-C_2(i,j)}{\max(C_2)}\right]\times 100
\end{equation}
\section{数字图像关联 (DIC)}

数字图像相关 (DIC) 是一种关键且广泛使用的非接触式方法来测量材料变形。 
近年来,在开发新颖的实验 DIC 方法和提高相关计算算法的性能方面取得了重大进展。 
因此,加密图像和原始图像的相同像素之间的关系如下所示:
\begin{equation}
    \text{NC}=\sum_{m}\sum_{n}\frac{(A_{mn}-\overline{A})(B_{mn}-\overline{B})}{\sqrt{(A_{mn}-\overline{A})^2+(B_{mn}-\overline{B})^2} }
\end{equation}
其中$A$和$B$分别表示原始图像和加密图像,以及它们的均值。 
较低的相关系数值是最佳的。