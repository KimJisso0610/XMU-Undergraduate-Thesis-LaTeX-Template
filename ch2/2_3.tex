\section{小波变换}

用于分析信号频率分量的一种有价值的仪器称为傅里叶变换。以傅立叶在整个时间轴上进行转换,
不可能确定增加特定频率的确切时刻。傅里叶变换和小波变换是相同的,具有完全不同的评价函数。
小波变换主要旨在仅通过变换时间扩展而不是形状来允许变化。两者的主要区别在于信号通过傅里
叶变换分解为余弦和正弦;然而,小波变换利用了傅里叶和实空间中的函数。通常,小波变换表述如下:
\begin{equation}
    f(a,b)=\int_{\infty}^{-\infty}f(x)\varPsi^*(a,b)\,\text{d}x \label{eq:wt}
\end{equation}
其中*代表复共轭符号,函数是一个函数,只要它遵循一定的规则,就可以任意选择。小波变换可以将信
号转换为时间、空间和频率作为独立的空间。它还侧重于任何局部细节的特定信号。因此,通过小波变换
可以有效地从信号中提取更多信息。\par
存在多种类型的小波变换用于特定目的。我们使用连续和离散小波变换从信号中提取更多信息。与傅里
叶变换类似,连续小波变换使用内积来测量信号和分析函数之间的相似性。理论分析是使用连续小波变
换的领域之一。在作为研究功能领域的计算机的特定实现中,必须对连续小波进行离散化。按照一些确
定的规则通过一组离散的小波尺度和平移运行小波变换称为离散小波变换。信号通过变换为相互正交的
小波群来分解,作为连续小波变换的必要变化。
此外,离散时间序列的实现有时被确定为离散时间连续小波变换。选择用于时频分解的小波是最重要的
一点。通过这种选择,我们可以影响结果的频率和时间分辨率。这种方式不能替代小波变换(WT)的基
本特征(低频具有错误的时间分辨率和真实频率;较高频率具有错误的频率分辨率和良好的时间)。然
而,以某种方式增加总时间分辨率的总频率是可能的。它与傅里叶和实空间中使用的小波宽度成正比。
使用 Morlet 小波,我们可以假设高频分辨率是频率中非常好的局部化小波。相反,利用高斯小波的导
数将导致正确的时间定位但频率较低。