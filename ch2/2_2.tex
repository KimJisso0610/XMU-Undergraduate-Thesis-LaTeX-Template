\section{基于逻辑图的混沌序列}

显示二次非线性的一个离散时间维非线性系统称为逻辑图。逻辑图函数可以表示为:
\begin{equation}
    f(x)=\mu_{x}(1-x) \label{eq:logic graph function 1}
\end{equation}
状态方程的形式表示为:
\begin{equation}
    x_{n+1}=f(x_{n})=\mu_{x_{n}}(1-x_{n}) \label{eq:logic graph function 2}
\end{equation}
其中$x_{n}\in (0,1)$和$\mu\in (0,4)$被称为控制参数或分岔参数。\par
这里,$x_{n}$表示系统在$n$时间的状态。$x_{n+1}$表示下一时刻状态,$n$表示离散时间。
通过重复迭代关闭,增加了一系列点$\{x_{n}\}_\infty$,称为轨道。
逻辑图的性能对$\mu$的值很敏感。对于$\mu=3.574$,逻辑图是混沌的。
对于不同的初级条件,使用两个逻辑图来执行重复操作。此外,动态测量两个逻辑图的状态值。通过这个操作,产生了混沌序列。
