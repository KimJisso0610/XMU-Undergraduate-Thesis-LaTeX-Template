\section{混沌学概述}

混沌理论最早是由麻省理工学院的数学家和气象学家Edward Lorenz在 60 年代初的天气预报实验中发现的。
该理论即将探索明显随机数据中的隐藏模式。它提供了一种方便的方法来解决自然和人工系统的非线性问题,
这些系统具有不可预测的行为,例如道路交通、股票市场、地震、健康心脏的节律、DNA 编码序列、天气和
气候条件。对初始条件高度敏感的系统可以在混沌理论的保护伞下进行研究,混沌理论有意提及蝴蝶效应。
蝴蝶效应通常被解释为蝴蝶在巴西拍打翅膀并在德克萨斯州引发飓风。这意味着大系统中的微小变化可能会产
生复杂的结果。在这种情况下,该系统可能是从天气模式到小行星运动或人们的互动的任何东西,整个系统受
到影响的微小变化。从科学上讲,它被称为对初始条件 Dooley 的敏感依赖。由于计算中的一些数值错误,产
生了各种初始条件。这些误差为某些动态系统提供了大相径庭的结果。这使得几乎不可能预测长期渲染的行为。
即使系统的行为是由同一系统的初始条件决定的,并且过程中不涉及随机元素,也会发生这种情况。具有这种
条件的动态系统称为确定性系统。不足以使它们可预测的这种确定性行为的动态系统被标记为确定性。
因此,Edward Lorenz 尝试用一个单一的定义来描述混沌理论的主要概念。他说:“现在可以决定未来,但
大概的现在不能决定大概的未来”。这种预测随机性问题是一个巨大的问题。