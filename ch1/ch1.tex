\chapter{引言}

近年来,图像加密一直是一个有吸引力的研究领域。它被广泛认为是一种用于安全传输的有用技术。
每个图像加密算法都旨在生成具有最高质量的嘈杂图像以保密信息。此外,图像加密对于保证网络上
的分类传输和图像容量具有较好的作用。随着互联网技术的飞速发展,数字通信变得更加广泛。人们
可以随时随地在互联网上发送数字图像。这导致了数字图像加密的发展。研究中表示数字图像加密的
不同方法与日益增加的安全必要性有关。基于混沌方法的图像加密是一种新颖的图像加密方法,它采
用随机混沌序列对图像进行加密,是解决高安全性和快速图像加密的棘手问题的有效途径。在过去的
几年里,出现了各种版本的混沌技术。目前,已采用四种方法进行图像加密,分别应用各种原理并实
现相同的目标。这四项原则包括共享和秘密分割、顺序排列、混沌动态系统和现代密码学,每项原则
都具有独特的特征。\citep{fu2012chaos}\par
本文的主要目的是提供一种基于混沌理论的数字图像加密新技术。它由 Y. Pourasad 等人提出。
然而,基于混沌的图像加密技术存在一些问题,包括准确性有限。为此,在本研究中,图像的加密
分为空间加密和变换域加密。在过去的几年里,一些图像加密方案被提出了频域和空间域。空间域
方法直接作用于普通图像的像素。\citep{yun2009digital, wang2008digital}由于这种方法包含高速加密,因此被广泛使用。使用变换域加密,
考虑到数字图像的一些典型属性,即高冗余和附近像素之间的强相关性。\par
本文以混沌序列和小波变换值以及图像加密算法的融合为导向。这种算法是通过分析算法来模拟的,
以发现差距。因此,算法得到了增强。该方法使用两个一维混沌系统,甚至可以使用基本非
线性方程来显示混沌行为。我们的主要目标,以及采用这种映射的比例,是发现一个新的离散时间
序列,与具有唯一参数的基本方程的逻辑映射的混沌输出相同。本文将在以下部分中介绍。在
“引言”部分,描述了问题的动机和陈述。此外,在“方法和材料”中,介绍了该方法的基本数学概念和
表达式。此外,在“提议的算法”部分,使用图形和表格描述了提议的模型实现的结果。此外,比较在
“算法”部分进行了介绍。最后,“结论”部分通过数值结果和透视概念总结了结果。\citep{yun2009digital, pourasad2021new}